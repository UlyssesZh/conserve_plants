\documentclass{article}
\usepackage{fancyhdr}
\usepackage{lastpage}
\usepackage{amsthm}
\usepackage{hyperref}

\newtheorem{assumption}{Assumption}

\pagestyle{fancy}
\fancyhf{}
\rhead{Page \thepage\ of \pageref{LastPage}}
\lhead{Team \# 10751}

\begin{document}

\begin{center}
Team Control Number

\Huge 10751

\normalsize ~

Problem Chosen

\Huge B

\Large 2020

HiMCM

Summary Sheet
\end{center}

\normalsize

Summary summary summary summary summary summary summary summary summary summary summary summary summary summary summary summary summary summary summary summary summary summary summary summary summary summary summary summary summary summary summary summary summary summary summary summary summary summary summary summary summary summary summary summary summary summary summary summary summary summary summary summary summary summary summary summary summary summary summary summary summary summary summary summary summary summary summary summary summary summary summary summary summary summary summary summary summary summary summary summary summary summary summary summary summary summary summary summary summary summary summary summary summary summary summary summary summary summary summary summary summary summary summary summary summary summary summary summary summary summary summary summary summary summary summary summary summary summary summary summary summary summary summary summary summary summary summary summary summary summary summary summary summary summary summary summary summary summary summary summary summary summary summary summary summary summary summary summary summary summary summary summary summary summary summary summary summary summary summary summary summary summary summary summary summary summary summary summary summary summary summary summary summary summary summary summary summary summary summary summary summary summary summary summary summary summary summary summary summary summary summary summary summary summary summary summary summary summary summary summary summary summary summary summary summary summary summary summary summary summary summary summary summary summary summary summary summary summary summary summary summary summary summary summary summary summary summary summary summary summary summary summary summary.

\newpage

\tableofcontents

\section{Introduction}
\label{sec:intro}

The decrease in biodiversity due to the extinction of endangered plants and animals is a serious problem.
To avoid this, people need to spend money on protecting the plants and animals.

However, in some places like Florida, greatly needing biodiversity conservation,
people do not have enough money to protect all of the imperiled species there.
They need to find out a proper scheme of funding and protecting to make full use of the funds to get as much benefit as possible.

Different projects of conservation have different cost, benefit, and required time,
as shown in the table of threatened plants data (abbreviated as TPD in the following parts of the paper).
It is to be determined which projects are selected in the optimal plan according to these factors.
After choosing the projects, when to start the projects should also be determined to balance the funds spent over time as possible.
To make it clear what is going to be done, the problem is restated mathematically in Section \ref{sec:restatement}.
The model and the method to derive the results are to be explained in Section \ref{sec:model}.

We will write a non-technical memo in Section \ref{sec:memo} to give our proposal according to our model.

\section{Variables}

\section{Restatement of the problem}
\label{sec:restatement}

There is a list of plant species to be protected.

The conservation of each species lasts several years, with different per-year costs during the process.
Our goal is, given a limited funding, to find out the best plan of conserving the species.

\section{General assumptions}

\begin{assumption}
Funds are provided once per year, depicted by $f_j$.
\end{assumption}

\section{The model and the approach}
\label{sec:model}

\section{Pros and cons}

Our model have some pros and cons in solving the problem described in \ref{sec:intro}.

The pros are
\begin{enumerate}
\item Pro 1.
\item Pro 2.
\item Pro 3.
\end{enumerate}

The cons are
\begin{enumerate}
\item Con 1.
\item Con 2.
\item Con 3.
\end{enumerate}

\newpage

\section{The memo}
\label{sec:memo}

Memo memo memo memo memo memo memo memo memo memo memo memo memo memo memo memo memo memo memo memo memo memo memo memo memo memo memo memo memo memo memo memo memo memo memo memo memo memo memo memo memo memo memo memo memo memo memo memo memo memo memo memo memo memo memo memo memo memo memo memo memo memo memo memo memo memo memo memo memo memo memo memo memo memo memo memo memo memo memo memo memo memo memo memo memo memo memo memo memo memo memo memo memo memo memo memo memo memo memo memo memo memo memo memo memo memo memo memo memo memo memo memo memo memo memo memo memo memo memo memo memo memo memo memo memo memo memo memo memo memo memo memo memo memo memo memo memo memo memo memo memo memo.

\newpage

\begin{thebibliography}{9}
\item Ref 1.
\end{thebibliography}

\end{document}